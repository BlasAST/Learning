\documentclass{article}
\usepackage{graphicx}
\usepackage[utf8]{inputenc}
\usepackage[spanish]{babel}
\usepackage{xcolor}
\usepackage{amsmath}
\usepackage{amssymb}
\usepackage{titling}

\setlength{\parindent}{0pt}
\pagecolor{black}
\color{white}
\title{
    % \includegraphics[width=10cm]{imgs/portada.jpg}
    \textbf{Representación y tratamiento de la información}
}
\author{BlasAST}
\date{Junio 2025}

\begin{document}
\maketitle

\newpage

\tableofcontents

\newpage

El objetivo de \textbf{la representación y el tratamiento de la información}
consiste en dotar a los usuarios de la capacidad de identificar principios,
conceptos o recursos elementales sobre los sistemas informáticos, ya sea a
nivel de hardware, software o la representación de datos. Además, se podrá
contrastar los sistemas y técnicas más adecuados para la representación
eficiente de un conjunto de datos.

\newpage

\section{Representación de los Números Enteros sin Signo}
Em nose, aqui lo que se muestra basicamente es porque usamos los
números como los usamos.

\subsection{Introducción a los Sistemas de Numeración}

Bueno, cuando contamos del 0 al 9 podemos darnos cuenta que usamos
solo 10 digitos, el resto de números que usamos estan formados
mediante la combinación de estos números. Esto de que usemos 10 digitos
(simbolos) es lo que se llama sistema de numeración.\\\\¿Qué es un sistema de
enumeración? Pues lo que usamos en nuestro día a día. Es como si en nuestra
cabeza tuvieramos un programa que nos permite diferencia que el 21 y el 12 no
son el mismo número, que expresan distinta cantidad, que ambos estan formados
por el 1 y el 2 pero que influye el orden. Eso es un sistema de numeración.
Un conjunto de simbolos finitos y unas reglas que utilizamos para expresar
distintas cantidades numéricas.\\\\\

La única preocupación que debemos de tener respecto a las restricciones del
sistema son las normas que trae con él, no nos limitan en cuanto a la cantidad
de simbolos ni en la forma en la que podemos usarlos.\\ Tampoco existen
restricciones que nos digan que reglas podemos aplicarles a los simbolos o
si son muchar reglas o pocas.\\\\

Todo esto viene porque podemos usar cualquier sistema de numeración que queramos.
Ejemplo:
\begin{quote}
    Si queremos representar cantidades numericas podríamos hacerlo con cualquier
    sistema como si solo utiliza 4 simbolos: \{0,1,2,3\} como si queremos que sean
    7 simbolos: \{1,a,5,S,:,9,0\} y ambos serían igual de validos. Con las reglas
    que le pongamos al sistema conseguimos que representen lo que queramos.\\
    Si quisieramos que el 0 seguido de un 3 sea el 81 podríamos hacerlo.
\end{quote}

Gracias a esta libertad existen multiples sistemas de numeración distintos entre
sí. Aunque, no todos serán tan prácticos o faciles de manejar como otros.\\\\

La cantidad de  combinaciones que se pueden formar depende del número total
de digitos que existan en el sistema. Esto es \textbf{la longitud de palabra}
que sería: 
$d$ cantidad de digitos, $n$ longitud de cadena y el resultado sería $d^n$ 
secuencias diferentes que se pueden construir.

Para poder saber a qué número en concreto representa la secuencia que hayamos
creado se utiliza la funcion \"Valor\" denotada mediante $V$. Esta función Valor
recibe como argumento un digito o varios y devuelve el resultado expresado en
número decimal.
Ejemplo:
\begin{quote}
    Si tomaramos el $V(xxy)$ y asumimos que xxy es el 76 el resultado sería:
    $V(xxy)=76$.
\end{quote}

Dado que cada simbolo necesita un valor surge el problema de memorizar
las asociaciones de los valores del sistema con el que trabajamos, por esto,
surgieron los sistemas posicionales y ponderados.

\subsection{Sistemas Posicionales y Ponderados}

Los sistemas que reciben este nombre son aquellos en los que cada digito
que conforma una cadena/secuencia este se ve afectado por su factor de escala
(segun la posición en la que esta), esto se llama \textbf{peso}.\\\\\
Por ejemplo:
\begin{quote}
    Si una cadena con X longitud lo podriamos ordenar asi:\\
    D = xyzi;\\
    D = dig3(x), dig2(y) ,dig1(z), dig0(i).
\end{quote}

Pues dependiendo de donde estan se toma dicho numero como la potencia,
siendo el de más de la izquierda el dígito más significativo y el de la más a la derecha
el menos significativo.\\\\

Al usar este método la funcionón valor deja de tener inconvenientes para los sistemas. Esta base que usamos
será lo que determine la solución del problema.
\subsection{Sistemas Relativos a una Base}
Un caso en particular de los sitemas posicionales y ponderados son los que usan como base del sistema
un número entero al que se llama \textbf{base} mediante el que se obtienen los \textbf{pesos} como
potencias de esta base. Los exponentes deben ser positivos dado que así evitamos la aparición de números
fraccionarios(por ahora).
\newpage
\subsubsection{Conjunto de valores representables(CVR) o rango}
Teniendo en cuenta que la base  $b$ es el número de números decimales respresentables y la longitud de
las cadenas es $n$, entonces el conjunto de valores que se pueden representar viene dado por $b^n$\\
Todos los valores representables se abarcan desde el más pequeño que se puede representar que estará 
formado por los digitos de menor valor como los más grandes formados por los de mayor valor. El rango
se calcula de esta forma:
\begin{quote}
    $CVR = [0, b^n - 1]$\\\\
    El rango siempre debe ser expresado en decimal.
\end{quote}

\subsubsection{Elección de la base}
Para saber que base escoger en un sistema de numeración no se puede elegir cualquier número, sino que debe
de tomarse \textbf{el número total de dígitos del sistema.}\\\\
Si $base > digitos$ el sistema sería incompleto por lo que habría números no representables en este sistema\\\\
Si $base < digitos$ sería redundante por lo que existiría múltiples representaciones de un mismo número.\\\\
Por eso:
\begin{itemize}
    \item $base = digitos$  el sistema es completo dado que puede representar todo lo necesario.
    \item Los digitos a utilizar en un sistema relativo a base tienen que elegirse sabiendo que el valor
    de  cualquiera de los digitos no supere a b.
    \item Para un mejor entendimiento deberemos de representar el Digito a representar y su base de esta forma:
    $D_b$
\end{itemize}
\newpage
\subsubsection{Sistemas de mayor implantación}
Los sistemas que más se suelen usar son los siguientes:
\begin{itemize}
    \item Binario: \\
    Base 2, digitos \{0, 1\}.\\
    Los pesos serían: $p_0 = 2^0 =1 ,\quad p_1 = 2^1 =2 ,\quad p_2 = 2^2 =4,
    \quad p_3 = 2^3 =8$, etc.
    \item Octal
    Base 8, digitos \{0,1,2,3,4,5,6,7\}\\
    Los pesos serían: $p_0 = 8^0 =1, \quad p_1 = 8^1 =8, \quad p_2 = 8^2 =64
   , \quad p_3 = 8^3 =512$, etc.
    \item Decimal
    Base 10, digitos \{0,1,2,3,4,5,6,7,8,9\}\\
    Los pesos serían: $p_0 = 10^0 =1, \quad p_1 = 10^1 =10, \quad p_2 = 10^2 =100
   , \quad p_3 = 10^3 =1000$, etc.
    \item Hexadecimal
    Base 16, digitos \{0,1,2,3,4,5,6,7,8,9,A,B,C,D,E,F\}. (A=10, B=11, etc)\\
    Los pesos serían: $p_0 = 16^0 =1, \quad p_1 = 16^1 =16, \quad p_2 = 16^2 =256$,
    etc.
\end{itemize}
\subsection{Conversiones entre Bases}
Para la conversión  entre bases hay distintas formas, a continuación se explica
brevemente.
\subsubsection{Conversión de base b a decimal}
Se aplica la funcion \"Valor\" matematicamente sería:
\begin{quote}
    $V(D) = \sum_{i = 0}^{n - 1} V(d_i)*b^i$
\end{quote}
\subsubsection{Conversión de decimal a base b}
Se realiza el método de las divisiones sucesivas:
 \begin{enumerate}
    \item Se divide el número decimal entre la base a la que se quiere convertir
    \item Si el cociente obtenido sigue siendo divisible entre la base se hacerlo
    recursivamente hasta obtener un cociente no divisible.
    \item Se usa el último cociente como dígito de mayor peso de la secuencia 
    resultante dada por ese último cociente y los restos en sentido inverso
    al obtenido.
 \end{enumerate}
\subsubsection{Conversión de base b a base c}
Para hacer esto no se puede hacer directamente. Lo que se hace es:
\begin{enumerate}
    \item Traducir la cadena de base b a decimal
    \item El resultado se convierte a la base c
\end{enumerate}

\section{Códigos Binarios}
En este apartado se profundiza más en el sistema de numeración binario, también
conocido como código binario natural. Es muy útil dado su sencillez y las 
ventajas que aporta a la realización de operaciones aritméticas con los
circuitos digitales que contienen los sistemas informáticos.\\\\

El motivo por el que se estudia el binario más a profundidad es porque no es
el único código de naturaleza binaria que existe dado que hay distintos sistemas
basados en el binario con sus características, particularidades, reglas y normas
La diferencia se basa en las distintas formas de emplear el 1 y el 0
para representar cantidades numéricas y la longitud de palabra que requiera.\\\\

Terminos necesarios:
\begin{itemize}
    \item \textbf{bit} designa cualquier digito de naturaleza binaria.
    \item  codificación para indicar la asignación de una cadena de bits
    a un número decimal o símbolo de nuestro alfabeto.
    \item Palabra codificada o secuencia binaria para nombrar a una cadena de bits
\end{itemize}

Entre los distintos tipos de sistemas binarios podemos clasificarlos en dos:
\begin{itemize}
    \item Ponderados $\rightarrow$ cada bit tiene un peso según su posición
    \item No ponderados $\rightarrow$ cada bit no tiene un peso asignado si no que
    se trata de una forma distintas.
\end{itemize}

\end{document}