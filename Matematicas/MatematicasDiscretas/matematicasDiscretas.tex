\documentclass{article}
\usepackage{graphicx} % Required for inserting images
\usepackage[utf8]{inputenc}
\usepackage[spanish]{babel}
\usepackage{xcolor}
\usepackage{amsmath}
\usepackage{amssymb}
\usepackage{titling}

\setlength{\parindent}{0pt}
\pagecolor{black}
\color{white}
\title{
    \includegraphics[width=10cm]{imgs/portada.jpg}
    \textbf{Matematicas Discretas}
}
\author{BlasAST}
\date{May 2025}

\begin{document}
\maketitle

\newpage

\tableofcontents

\newpage

\section{Introducción a las matemáticas discretas}
Mátematicas Discretas no significa que sea discreta en si, ni que sea reservada o prudente.

Son las que se encargan de estudiar estructuras y objetos matemáticos.
Para que se entienda un poco más es como en programación.
Hay distintas formas de guardar los datos que se
quieren guardar y hay distintos objetos o cosas que podemos guardar en esas formas de almacenamiento.\\

Algunas de las cosas que se estudian en este ámbito son las siguientes:

\begin{itemize}
    \item Conteo y combinatoria que consiste en enumerar procesos y organizarlo,
    combinarlo, seleccionar algunos elementos concretos, etc.
    \item Descubrir patrones/estructuras
    \item Entender la complejidad de los objetos/elementos con los que tratamos.
    \item Realizar demostraciones matemáticas basandose en hechos o demostrar correciones
    o propiedades de algoritmos.
    \item Conjuntos, relaciones, funciones, etc.
    \item Teoría de números (estudio de números enteros y sus propiedades)
    \item Grafos que tratan de modelar relaciones entre elementos como podrían ser
    redes de comunicación, carreteras, etc. Algunos de los algoritmos de grafos
    como la búsqueda del camino más corto entre dos lugares, etc.
\end{itemize}
\newpage
\section{Teoría de conjuntos}
Bien, para empezar vamos a analizar que son las estructuras.\\

Una estructura es una agrupación de objetos y los objetos son cualquier tipo
de dato.
Lo que hacen las Matemáticas Discretas es estudiar dichas estructuras.\\
Hay distintos tipos de estructuras pero los más comunes suelen ser:
\textbf{conjuntos, combinaciones, relaciones y grafos}.\\\\

\subsection{Conjuntos}

Los conjuntos son colecciones (como una estantería) de distintos
objetos/elementos/miembros del conjunto que no están ordenados.\\

Estos objetos se listan mediante los simbolos \textbf{"\{\}"}
separando los elementos por \textbf{","}

\begin{quote}
    Un ejemplo sería:\\
    $ A = \{x, y, z\} \leftarrow$  \text{Cada elemento solo aparece una vez}\\
    $\uparrow $\\
    $\qquad \text{Este el nombre de la estructura}$ \\
\end{quote}

Un conjunto podría tener 0,1,n, o infinitos elementos.
Algo importante de entender de los conjuntos es el la idea de \textbf{pertenencia}.
Un elemento pertenece ($\in $) a un conjunto.\\

\textbf{$x \in A \quad$} $\leftarrow  x$ pertenece a $A$ por lo que $x$ es un elemento
de $A$ y entonces podemos encontrar a $x$ dentro de $A$\\

Si un elemento no esta contenido en el conjunto se dice que \textbf{$x \notin A$}\\

\begin{quote}
    Ejemplo:
    $K = \{1, 8, A, :), L\}$\\\\
    \begin{itemize}
        \item $8 \in K$\\
        \item $:) \in K$\\
        \item $:( \notin K)$\\
    \end{itemize}
\end{quote}
Un conjunto vacio (sin elementos) se denota con $ \emptyset \quad o \quad \{\}$.\\
El conjunto universal se denota con $ U \quad o \quad \Omega$

\subsection{Cardinalidad}
La cardinalidad consiste en saber el número de elementos de A. Se denota por $|A|$
 Si $A \neq \varnothing$ y es finito, entonces $|A| \in \mathbb{N}$ (números naturales).

 \begin{quote}
    Ejemplo:\\

    $A = \{x,y,z\}$   $|A| = 3$ \\\\
    $B = \{a,e,i,o,u\}$   $|B| = 5$ \\\\
    $C = \{2,3,5,7,11,13,17,19,23,29\}$ (10 primeros primos) $|C| = 10$\\\\
    $D = \{$ :) , :( , :D , :/ $\}$   $|D| = 4$
 \end{quote}

\section{Representación de conjuntos}

Los conjuntos se pueden representar de dos formas distintas.

\begin{itemize}
    \item Extensión. Lista de todos los elementos del conjunto.
    \item Compresión Notación constructora para su representación.
\end{itemize}

\begin{quote}
    Ejemplo de extensión:
        De la forma que estabamos viendo. Representar cada elemento entre
        corchetes.\\\\
        $\{1,2,3,4,5,A,V,B,a,s,h\}$\\\\
    Ejemplo por compresión:
        Por ejemplo los números primos de antes, en vez de 10 si quisieramos
        mostrar todos sería:\\\\
        $P = \{x \in \mathbb{N} es primo\}$
\end{quote}


\end{document}